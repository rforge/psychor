%\VignetteIndexEntry{anacor} 

\documentclass[article]{Z}
\usepackage{amsmath, amsfonts, thumbpdf}
\usepackage{float,amssymb}
\usepackage{hyperref}
\usepackage{Sweave}

\newcommand{\defi}{\mathop{=}\limits^{\Delta}}  
%%%%%%%%%%%%%%%%%%%%%%%%%%%%%%
%% declarations for jss.cls %%%%%%%%%%%%%%%%%%%%%%%%%%%%%%%%%%%%%%%%%%
%%%%%%%%%%%%%%%%%%%%%%%%%%%%%%

%% almost as usual
\author{Jan de Leeuw\\University of California, Los Angeles \And 
        Patrick Mair \\Wirtschaftsuniversit\"at Wien}
\title{Homogeneity Analysis in R:\\ The Package \pkg{homals}}
%
%%% for pretty printing and a nice hypersummary also set:
\Plainauthor{Jan de Leeuw, Patrick Mair} %% comma-separated
\Plaintitle{Homogeneity Analysis in R:\\ The Package homals} %% without formatting
\Shorttitle{Homals in \proglang{R}} %% a short title (if necessary)

%% an abstract and keywords
\Abstract{
Homogeneity analysis combines maximizing the correlations between variables of a multivariate data set with
that of optimal scaling. In this article we present methodological and practical issues of the \proglang{R} package \pkg{homals} which performs homogeneity analysis and various extensions. By setting rank constraints nonlinear principal component analysis can be performed. The variables can be partitioned into sets such that homogeneity analysis is extended to nonlinear canonical correlation analysis or to predictive models which emulate discriminant analysis and regression models. For each model the scale level of the variables can be taken into account by setting level constraints. All algorithms allow for missing values.
}
\Keywords{homogeneity analysis, correspondence analysis, nonlinear principal component analysis, nonlinear canonical correlation analysis, homals, \proglang{R}}
\Plainkeywords{homogeneity analysis, correspondence analysis, nonlinear principal component analysis, nonlinear canonical correlation analysis, homals, R} %% without formatting
%% at least one keyword must be supplied

%% publication information
%% NOTE: This needs to filled out ONLY IF THE PAPER WAS ACCEPTED.
%% If it was not (yet) accepted, leave them commented.
%% \Volume{13}
%% \Issue{9}
%% \Month{September}
%% \Year{2004}
%% \Submitdate{2004-09-29}
%% \Acceptdate{2004-09-29}

%% The address of (at least) one author should be given
%% in the following format:
%%\Address{
%%  Jan de Leeuw\\
%% Department of Statistics\\
%%  University of California, Los Angeles\\
%%  E-mail: \email{deleeuw@stat.ucla.edu}\\
%%  URL: \url{http://www.stat.ucla.edu/~deleeuw/}
%%}
%% It is also possible to add a telephone and fax number
%% before the e-mail in the following format:
%% Telephone: +43/1/31336-5053
%% Fax: +43/1/31336-734

%% for those who use Sweave please include the following line (with % symbols):
%% need no \usepackage{Sweave.sty}

%% end of declarations %%%%%%%%%%%%%%%%%%%%%%%%%%%%%%%%%%%%%%%%%%%%%%%


\begin{document}
\section{Introduction}
\label{sec:int}
During the last years correspondence analysis (CA) has become a popular descriptive statistical method to analyze categorical data \citep{Benzecri:73, Greenacre:84, Gifi:90, Greenacre+Blasius:06}. Due to the fact that the visualization capabilities of statistical software have increased during this time, researchers of many areas apply CA and map objects and variables (and their respective categories) onto a common metric plane. 
   
Currently, \proglang{R} \citep{R:07} offers a variety of routines to compute CA and related models. An overview of corresponding functions and packages is given in \citet{Mair+Hatzinger:07}. The package \pkg{ca} \citep{Nenadic+Greenacre:06} is a comprehensive tool to perform simple and multiple CA. Various two- and three-dimensional plot options are provided.  

In this paper we present the \proglang{R} package \pkg{homals}, starting from the simple homogeneity analysis, which corresponds to a multiple CA, and providing several extensions. \citet{Gifi:90} points out that homogeneity analysis can be used in a \emph{strict} and a \emph{broad} sense. In a strict sense homogeneity analysis is used for the analysis of strictly categorical data, with a particular loss function and a particular algorithm for finding the optimal solution. In a broad sense homogeneity analysis refers to a class of criteria for analyzing multivariate data in general, sharing the characteristic aim of optimizing the homogeneity of variables under various forms of manipulation and simplification \citep[p.81]{Gifi:90}. This view of homogeneity analysis will be used in this article since \pkg{homals} allows for such general computations. Furthermore, the two-dimensional as well as three-dimensional plotting devices offered by \proglang{R} are used to develop a variety of customizable visualization techniques.    
More detailed methodological descriptions can be found in \citet{Gifi:90} and some of them are revisited in \citet{Michailidis+deLeeuw:98}. 


\section{Homogeneity Analysis}
In this section we will focus on the underlying methodological aspects of \pkg{homals}. Starting with the formulation of the loss function, the classical alternating least squares algorithm is presented in brief and the relation to CA is shown. Starting from basic homogeneity analysis we elaborate various extensions such as nonlinear canonical analysis and nonlinear principal component analysis. 

\subsection{Establishing the loss function}
Homogeneity analysis is based on the criterion of minimizing the departure from homogeneity. Homogeneity is measured by a loss function. To write the corresponding basic equations the following definitions are needed. For $i=1,\ldots,n$ objects, data on $m$ (categorical) variables are collected where each of the $j=1,\ldots,m$ variable takes on $k_j$ different values (their \emph{levels} or \emph{categories}). We code them using $n\times k_j$ binary \emph{indicator matrices} $G_j$, i.e., a dummy matrix for each variable. The whole set of indicator matrices can be collected in a block matrix
\begin{equation}
G\defi\begin{bmatrix}G_1&\vdots&G_2&\vdots&\cdots&\vdots&G_m\end{bmatrix}.
\end{equation}
Missing observations are coded as complete zero row sums; if object $i$ is missing on variable $j$, then row
sum $i$ of $G_j$ is 0, otherwise row sum becomes 1 since the category entries are disjoint. This corresponds to the first missing option presented in \citet[p.74]{Gifi:90}. Other possibilities would be to add an additional column to the indicator matrix for each variable with missing data or to add as many additional columns as there are missing data for the $j$-th variable. However, all row sums of $G_j$ are collected in the diagonal matrix $M_j$. Suppose $M_\star$ is the sum of the $M_j$ and $M_\bullet$ is their average. Furthermore, we define
\begin{equation}
D_j^{}\defi G_j'M_j^{}G_j^{}=G_j'G_j^{},
\end{equation}
where $D_j$ is the diagonal matrix ($k_j\times k_j$) with the relative frequencies of variable $j$ in its main diagonal. 

Now let $X$ be the unknown $n\times p$ matrix containing the coordinates (\emph{object scores}) of the object projections into $\mathbb{R}^p$. Furthermore, let $Y_j$ be the unknown $k_j \times p$ matrix containing the coordinates of the category projections into the same $p$-dimensional space (\emph{category quantifications}). 
The problem of finding these solutions can be formulated by means of the following loss function to be minimized:
\begin{equation}
\label{eq:loss}
\sigma(X;Y_1,\ldots,Y_m)\defi\frac{1}{m}\sum_{j=1}^m\mathbf{tr}(X-G_jY_j)'M_j(X-G_jY_j)
\end{equation}
We use the normalization $\mathbf{u}'M_\bullet X=0$ and $X'M_\bullet X=I$ in order to avoid the trivial solution $X=0$ and $Y_j=0$. The first restriction centers the graph plot (see Section \ref{sec:R}) around the origin whereas the second restriction makes the columns of the object score matrix orthogonal.

\subsection{Geometry of the loss function}
In the \pkg{homals} package we use homogeneity analysis as graphical method to explore multivariate data sets. The \emph{joint plot} where the object scores and the category quantifications are mapped in a joint space, can be considered as the classical or standard homals plot. The category points are the center of gravity of the object points that share the same category. The larger the spread between category points the better a variable discriminates and thus, it indicates how much a variable contributes to relative loss. The distance between two object scores is related to the ``similarity'' between their response patterns. A ``perfect'' solution, i.e., without any loss at all, would imply that all object points coincide with their category points.

Moreover, we can think of $G$ as the adjacency matrix of a bipartite graph in which the
$n$ objects and the categories $k_j$ are the vertices. In the corresponding \emph{graph plot} an object and a category are connected by an edge if the object is in the corresponding category. The loss in (\ref{eq:loss}) pertains to the sum of squares of the line lengths in the graph plot. Producing a \emph{star plot}, i.e., connecting the object scores with their category centroid, the loss corresponds to the sum over variables of the sum of squared line lengths. More detailed plot descriptions are given in Section \ref{sec:R}.


\subsection{Minimizing the loss function}
Typically, the minimization problem is solved by the iterative \emph{alternating least squares algorithm} (ALS; sometimes quoted as \emph{reciprocal averaging algorithm}). At iteration $t=0$ we start with arbitrary object scores $X^{(0)}$. Each iteration $t$ consists of three steps:
\begin{enumerate}
\item Update category quantifications: $Y_j^{(t)}=D_j^{-1}G_j'X^{(t)}$
\item Update object scores: $\tilde X^{(t)}=M_\star^{-1}\sum_{j=1}^m G_j^{}Y_j^{(t)}$
\item Normalization: $X^{(t+1)}=\mathbf{orth}(\tilde X^{(t)})$
\end{enumerate}
Note that matrix multiplications using indicator matrices can be implemented efficiently as cumulating the sums of rows over $X$ and $Y$. 

Here $\mathbf{orth}$ is some technique which computes an orthonormal basis for the column space of a matrix. We can use QR decomposition, modified Gram-Schmidt, or the singular value decomposition (SVD). In \pkg{homals} the left singular vectors of $\tilde X^{(k)}$, here denoted as $\mathbf{lsvec}$, are used. 

To simplify, let $P_j$ denote the orthogonal projector on the subspace spanned by the columns of $G_j$, i.e.,
$P_j^{}=G_j^{}D_j^{-1}G_j'$. Correspondingly, the sum over the $m$ projectors is 
\begin{equation}
P_\star=\sum_{j=1}^m P_j = \sum_{j=1}^m G_j^{}D_j^{-1}G_j'.
\end{equation}
Again, $P_\bullet$ denotes the average. By means of the $\mathbf{lsvec}$ notation and including $P_\bullet$ we can describe a complete iteration step as 
\begin{equation}
\label{eq:Xlsv}
X^{(k+1)}=\mathbf{lsvec}(\tilde X^{(k)})=\mathbf{lsvec}(M_\bullet^{-1}P_\bullet^{} X^{(k)}).
\end{equation}
In each iteration we compute the value of the loss function to monitor convergence. Note that Formula (\ref{eq:Xlsv}) is not suitable for computation, because it replaces computation with sparse indicator matrices by computations with a dense average projector. 

Computing the \pkg{homals} solution in this way is the same as performing a CA on $G$. Usually, multiple CA solves the generalized eigenproblem for the Burt matrix $C=G'G$ and its diagonal \(D\) \citep{Greenacre:84,Greenacre+Blasius:06}.
Thus, we can put the problem in Equation \ref{eq:loss} into a SVD context \citep{deLeeuw+Michailides+Wang:99}. Using the block matrix notation, we have to solve the generalized singular value problem of the form
\begin{eqnarray}
GY=M_\star X\Lambda, \\
G'X=DY\Lambda,
\end{eqnarray}
or equivalently one of the two generalized eigenvalue problems
\begin{eqnarray}
GD^{-1}G'X=M_\star X\Lambda^2, \\
G'M_\star^{-1}GY=DY\Lambda^2.
\end{eqnarray}
Here the eigenvalues $\Lambda^2$ are the ratios along each dimension of the average between-category variance and the average total variance.
Also $X'P_jX$ is the between-category dispersion for variable $j$. Further elaborations can be found in \citet{Michailidis+deLeeuw:98}. 

Compared to the classical SVD approach, the ALS algorithm only computes the first $p$ dimensions of the solution. This leads to an increase in computational efficiency. Moreover, by capitalizing the sparseness of $G$, \pkg{homals} is able to handle large data sets.

\section{Extensions of homogeneity analysis}
\citet{Gifi:90} provides various extensions of homogeneity analysis and elaborates connections to other multivariate methods. The package \pkg{homals} allows for imposing restrictions on the variable ranks and levels as well as defining sets of variables. These options offer a wide spectrum of additional possibilities for multivariate data analysis beyond classical homogeneity analysis (cf. broad sense view in the Introduction).

\subsection{Nonlinear principal component analysis}
Having a $n \times m$ data matrix with metric variables, principal components analysis (PCA) is a common technique to reduce the dimensionality of the data set, i.e., to project the variables into a subspace $\mathbb{R}^p$ where $p \ll m$. The Eckart-Young theorem states that this classical form of \emph{linear} PCA can be formulated by means of a loss function. Its minimization leads to a $n\times p$ matrix of \emph{component scores} and an $m \times p$ matrix of \emph{component loadings}.

However, having nonmetric variables, nonlinear PCA (NLPCA) can be used. The term ``nonlinear" pertains to nonlinear transformations of the observed variables \citep{deLeeuw:06}. In Gifi terminology, NLPCA can be defined as homogeneity analysis with restrictions on the quantification matrix $Y_j$. Let us denote $r_j \leq p$ as the parameter for the imposed restriction on variable $j$. If no restrictions are imposed, as e.g. for a simple homals solution, 
$r_j = k_j -1$ iff $k_j \leq p$, and $r_j = p$ otherwise. 

We start our explanations with the simple case for $r_j = 1$ for all $j$. In this case we say that all variables are \emph{single} and the rank restrictions are imposed by 
\begin{equation}
\label{eq:srank}
Y_j = \mathbf{z}_j\mathbf{a}_j',
\end{equation}
where $\mathbf{z}_j$ is a vector of length $k_j$ with category quantifications and $\boldsymbol{a}_j$ a vector of length $p$ with weights. Thus, each quantification matrix is restricted to rank-1, which allows for the existence of object scores with a single category quantification. 

Straightforwardly, Equation \ref{eq:srank} can be extended to the general case 
\begin{equation}
\label{E:res}
Y_j=Z_jA_j'
\end{equation}
where again $1 \leq r_j \leq \min{(k_j-1,p)}$, $Z_j$ is $k_j \times r_j$ and $A_j$ is $p \times r_j$. We require, without loss of generality, that
$Z_j'D_j^{}Z_j^{}=I$. Thus, we have the situation of \emph{multiple quantifications} which implies imposing an additional constraint each time PCA is carried out. 

To establish the loss function for the rank constrained version we write $r_\star$ for the sum of the $r_j$ and $r_\bullet$ for their average. The block matrix $G$ of dummies now becomes 
\begin{equation}
Q\defi\begin{bmatrix}G_1Z_1&\vdots&G_2Z_2&\vdots&\cdots&\vdots&G_mZ_m\end{bmatrix}.
\end{equation}
Gathering the $A_j$'s in a block matrix as well, the $p \times r_\star$ matrix
\begin{equation}
A\defi\begin{bmatrix}A_1&\vdots&A_2&\vdots&\cdots&\vdots&A_m\end{bmatrix}
\end{equation}
results. Then, Equation \ref{eq:loss} becomes
\begin{eqnarray}
\label{eq:lossr}
\sigma(X;Z;A)=
\sum_{j=1}^m\mathbf{tr}(X-G_jZ_j^{}A_j')'M_j(X-G_jZ_j^{}A_j')=\nonumber \\
=\mathbf{tr}(Q-XA')'(Q-XA')+m(p-r_\bullet).
\end{eqnarray}
This shows that $\sigma(X;Y_1,\cdots,Y_m)\geq m(p-r_\bullet)$ and the loss is equal to
this lower bound if we can choose the $Z_j$ such that $Q$ is of rank $p$. In fact, by
minimizing (\ref{eq:lossr}) over $X$ and $A$ we see that
\begin{equation}
\sigma(Z)\defi\min_{X,A}\sigma(X;Z;A)=
\sum_{s=p+1}^{r_\star}\lambda_s^2(Z)+m(p-r_\bullet),
\end{equation}
where the $\lambda_s$ are the ordered singular values. A corresponding example in terms of a \emph{lossplot} is given in Section \ref{sec:R}.
Now we will take into account the scale level of the variables in terms of restrictions within $Z_j$. To do this, the starting point is to split up Equation \ref{eq:lossr} into two separate terms \citep{Gifi:90,Michailidis+deLeeuw:98}. Using \(\hat Y_j=D_j^{-1}G_j'X\) this leads to 
\begin{eqnarray}
 & \sum_{j=1}^m \mathbf{tr}(X-G_jY_j)'M_j(X-G_jY_j) & \nonumber \\
& = \sum_{j=1}^m \mathbf{tr}(X-G_j(\hat Y_j + (Y_j - \hat Y_j)))'M_j(X-G_j(\hat Y_j + (Y_j - \hat Y_j))) & \nonumber \\
= & \sum_{j=1}^m\mathbf{tr}(X-G_j\hat Y_j)'M_j(X-G_j\hat Y_j)+\sum_{j=1}^m \mathbf{tr}(Y_j-\hat Y_j)'D_j(Y_j-\hat Y_j). &
\end{eqnarray}
Obviously, the rank restrictions $Y_j=Z_jA_j'$ affect only the second term and hence, we will proceed on our explanations by regarding this particular term only, i.e.,
\begin{equation}
\label{eq:lt2}
\sigma(Z;A)=\sum_{j=1}^m \mathbf{tr}(Z_jA_j'-\hat Y_j)'D_j(Z_jA_j'-\hat Y_j).
\end{equation}
Now, level constraints for nominal, ordinal, polynomial, and numerical variables can be imposed on $Z_j$ in the following manner. 
For nominal variables, all columns in $Z_j$ are unrestricted. Equation \ref{eq:lt2} is minimized under the conditions $\mathbf{u}'D_jZ_j=0$, $Z_j'D_jZ_j=I$, and $\mathbf{u}'D_jY_j=0$. The stationary equations are 
\begin{subequations}
\begin{align}
A_j&=Y_j'D_jZ_j,\\
Y_jA_j&=Z_jW+\mathbf{uh}',
\end{align}
\end{subequations}
with $W$ as a symmetric matrix of Langrange multipliers. Solving, we find
\begin{equation}
\mathbf{h}=\frac{1}{\mathbf{u}'D_j\mathbf{u}}A_j'Y_j'D_j\mathbf{u}=\mathbf{0},
\end{equation}
and thus, letting $\overline Z_j\defi D_j^{1/2}Z_j$ and $\overline Y_j\defi D_j^{1/2}Y_j$, it follows that
\begin{equation}
\overline Y_j\overline Y_j' \overline Z_j=\overline Z_jW.
\end{equation}
If $\overline Y_j=K\Lambda L'$ is the SVD of $\overline Y_j$, then
we see that $\overline Z_j=K_rO$ with $O$ an arbitrary rotation matrix. Thus, $Z_j=D_j^{-1/2}K_rO$, and
$A_j=\overline Y_j'\overline Z_j=L_r\Lambda_r O$. Moreover, $Z_jA_j'=D_j^{-1/2}K_r\Lambda_rL_r'$.

Having ordinal variables, the first column of $Z_j$ is constrained to be either increasing or decreasing, the rest is free. Again (\ref{eq:lt2}) has to be minimized under the condition $Z_j'D_j^{}Z_j^{}=I$ (and optionally additional conditions on $Z_j$). If we minimize over $A_j$, we can also solve the problem $\mathbf{tr}(Z_j'D_j^{}Y_j^{}Y_j'D_j^{}Z_j^{})$ with $Z_j'D_j^{}Z_j^{}=I$.

For polynomial constraints the matrix $Z_j$ are the first \(r_j\) orthogonal polynomials. Thus all \(p\) columns of \(Y_j\) are polynomials of degree \(r_j\).  In the case of numerical variables, the first column in $Z_j$ denoted by $\mathbf{z}_{j0}$ is fixed and linear with the category numbers, the rest is free. 
Hence, the loss function in (\ref{eq:lt2}) changes to 
\begin{equation}
\label{eq:lossnum}
\sigma(Z,A)=\sum_{j=1}^m \mathbf{tr}(Z_jA_j'+\mathbf{z}_{j0}\mathbf{a}_{j0}'- \hat Y_j)'D_j(Z_jA_j'+\mathbf{z}_{j0}\mathbf{a}_{j0}'-\hat Y_j).
\end{equation}
Since column $\mathbf{z}_{j0}$ is fixed, $Z_j$ is a $k_j \times (r_j-1)$ matrix and $A_j$, with $\mathbf{a}_{j0}$ as the first column, is $p \times (r_j-1)$. In order to minimize (\ref{eq:lossnum}), $\mathbf{z}_{j0}'D_jZ_j=0$ is required as minimization condition.

Note that level constraints can be imposed additionally to rank constraints. If the data set has variables with different scale levels, \pkg{homals} allows for setting level constraints for each variable $j$ separately. 

\subsection{Nonlinear canonical correlation analysis}
In Gifi terminology, nonlinear canonical correlation analysis (NLCCA) is called ``OVERALS" \citep{vanderBurg+deLeeuw+Verdegaal:88, vanderBurg+deLeeuw+Dijksterhuis:94}. This is due to the fact that it has most of the other Gifi-models as special cases. 
In this section the relation to homogeneity analysis is shown. The \pkg{homals} package allows for the definition of \emph{sets} of variables and thus, for the computation NLCCA between $g = 1,\ldots,K$ sets of variables. 

Recall that the aim of homogeneity analysis is to find $p$ orthogonal vectors in $m$ indicator matrices $G_j$. 
This approach can be extended in order to compute $p$ orthogonal vectors in $K$ general matrices $G_v$, each of dimension $n \times m_v$ where $m_v$ is the number of variables ($j = 1,\ldots ,m_v$) in set $v$. Thus, 
\begin{equation}
G_v\defi\begin{bmatrix}G_{v_1}&\vdots&G_{v_2}&\vdots&\cdots&\vdots&G_{v_{m_v}}\end{bmatrix}.
\end{equation}
The loss function can be stated as
\begin{equation}
\label{eq:lcca}
\sigma(X;Y_1,\ldots,Y_K)\defi\frac{1}{K}\sum_{v=1}^K\mathbf{tr}\left(X-\sum_{j=1}^{m_v}G_{v_j}Y_{v_j}\right)'M_v\left(X-\sum_{j=1}^{m_v}G_{v_j}Y_{v_j}\right).
\end{equation}
$X$ is the $n \times p$ matrix with object scores, $G_{v_j}$ is $n \times k_j$, and $Y_{v_j}$ is the $k_j \times p$ matrix containing the coordinates. Missing values are taken into account in $M_v$ which is the element-wise minimum of the $M_j$ in set $v$. The normalization conditions are $XM_\bullet X = I$ and $\mathbf{u}'M_\bullet X=0$ where $M_\bullet$ is the average of $M_v$.

Since NLPCA can be considered as special case of NLCCA, i.e., for $K=m$, all the additional restrictions for different scaling levels can straightforwardly be applied for NLCCA. 
Unlike classical canonical correlation analysis, NLCCA is not restricted to two sets of variables but allows for the definition of an arbitrary number of sets. Furthermore, if the sets are treated in an asymmetric manner predictive models such as regression analysis and discriminant analysis can be emulated. For $v=1,2$ sets this implies that $G_1$ is $n \times 1$ and $G_2$ is $n \times m-1$. Corresponding examples will be given in Section \ref{sec:pmcca}.

\subsection{Cone restricted SVD}
In this final methodological section we show how the loss functions of these models can be solved in terms of cone restricted SVD. All the transformations discussed above are projections on some convex cone $\mathcal{K}_j$.  
For the sake of simplicity we drop the $j$ and $v$ indexes and we look only at the second term of the partitioned loss function (see Equation \ref{eq:lt2}), i.e., 
\begin{equation}
\sigma(Z,A)=\mathbf{tr}(ZA'-\hat Y)'D(ZA'- \hat Y),
\end{equation}
over $Z$ and $A$, where $\hat Y$ is $k\times p$, $Z$ is $k\times r$, and $A$ is $p\times r$. Moreover
the first column $z_0$ of $Z$ is restricted by $z_0\in\mathcal{K}$, with $\mathcal{K}$
as a convex cone. $Z$ should also satisfy the common normalization conditions $u'DZ=0$ and $Z'DZ=I$.

The basic idea of the algorithm is to apply alternating least squares with rescaling.
Thus we alternate minimizing over $Z$ for fixed $A$ and over $A$ for fixed $Z$.
The ``non-standard" part of the algorithm is that we do not impose the normalization
conditions if we minimize over $Z$. We show below that we can still produce a
sequence of normalized solutions with a non-increasing sequence of loss function
values.

Suppose $(\hat Z,\hat A)$ is our current best solution. To improve it we  first minimize over the non-normalized $Z$, satisfying the cone constraint, and keeping $A$ fixed at $\hat A$. This gives $\tilde Z$ and a corresponding loss function value $\sigma(\tilde Z,\hat A)$. Clearly,
\begin{equation}
\sigma(\tilde Z,\hat A)\leq\sigma(\hat Z,\hat A),
\end{equation}
but $\tilde Z$ is not normalized. Now update $Z$ to $Z^+$ using the weighted Gram-Schmidt solution $\tilde Z=Z^+S$ for $Z$ where $S$ is the Gram-Schmidt triangular matrix. The first column $\tilde z_0$ of $\tilde Z$ satisfies the cone constraint, and because of the nature of Gram-Schmidt, so does the first column of $Z^+$. Observe that it is quite possible that
\begin{equation}
\sigma(Z^+,\hat A)>\sigma(\hat Z,\hat A).
\end{equation}
This seems to invalidate the usual convergence proof, which is based on
a non-increasing sequence of loss function values. But now also adjust
$\hat A$ to $\overline A=\hat A(S^{-1})'$. Then $\tilde Z\hat A'=Z^+\overline A'$, and thus
\begin{equation}
\sigma(\tilde Z,\hat A)=\sigma(Z^+,\overline A).
\end{equation}
Finally compute $A^+$ by minimizing $\sigma(Z^+,A)$ over $A$. Since $\sigma(Z^+,A^+)\leq\sigma(Z^+,\overline A)$ 
we have the chain
\begin{equation}
\sigma(Z^+,A^+)\leq\sigma(Z^+,\overline A)=\sigma(\tilde Z,\hat A)\leq\sigma(\hat Z,\hat A).
\end{equation}
In any iteration the loss function does not increase. In actual computation, it is not necessary to compute 
$\overline A$, and thus it also is not necessary to compute the Gram-Schmidt triangular matrix $S$.

\section[The R package homals]{The \proglang{R} package \pkg{homals}}
\label{sec:R}
At this point we show how the models described in the sections above can be computed using the package \pkg{homals} in \proglang{R} \citep{R:07} available on \href{http://cran.r-project.org}{CRAN}.

The core routine of the package is \code{homals}. The extended models can be fitted through appropriate settings on the parameters \code{sets}, \code{rank}, and \code{level}. An object of class \texttt{"homals"} is created and the following methods are provided: \code{print}, \code{summary}, \code{plot}, \code{plot3d}, \code{scatterplot3d} and \code{predict}. 

The package offers a wide variety of plots; some of them are discussed in \citet{Michailidis+deLeeuw:98} and \citet{ Michailidis+deLeeuw:01}. In the \code{plot} method the user can specify the type of plot through the argument \code{plot.type}. For some plot types three-dimensional versions are provided in \code{plot3d} (dynamic) and \code{plot3dstatic}:
\begin{itemize}
\item Object plot (\texttt{"objplot"}): Plots the scores of the objects (rows in data set) on two or three dimensions. 
\item Category plot (\texttt{"catplot"}): Plots the rank-restricted category quantifications for each variable separately. Three-dimensional plot is available.
\item Voronoi plot (\texttt{"vorplot"}): Produces a category plot with Voronoi regions.
\item Joint plot (\texttt{"jointplot"}): The object scores and category quantifications are mapped in the same (two- or three-dimensional) device.
\item Graph plot (\texttt{"graphplot"}): Basically, a joint plot is produced with additional connections between the objects and the corresponding response categories.
\item Hull plot (\texttt{"hullplot"}): For each single variable the object scores are mapped onto two dimensions and the convex hull for each response category is drawn.
\item Label plot (\texttt{"labplot"}): Similar to object plot, the object scores are plotted but for each variable separately with the corresponding category labels. A three-dimensional version is provided.
\item Span plot (\texttt{"spanplot"}): Like label plot, it maps the object scores for each variable and it connects them by the shortest path within each response category.
\item Star plot (\texttt{"starplot"}): Again, the object scores are mapped on two or three dimensions. In addition, these points are connected with the category centroid. 
\item Loss plot (\texttt{"lossplot"}): Plots the rank-restricted category quantifications against the unrestricted for each variable separately.
\item Projection plot (\texttt{"prjplot"}): For variables of rank 1 the object scores (two-dimensional) are projected onto a straight line determined by the rank restricted category quantifications.
\item Vector plot (\texttt{"vecplot"}): For variables of rank 1 the object scores (two-dimensional) are projected onto
a straight line determined by the rank restricted category quantifications.  
\item Transformation plot (\texttt{"trfplot"}): Plots variable-wise the original (categorical) scale against the transformed (metric) scale $Z_j$ for each solution.
\item Loadings plot (\texttt{"loadplot"}): Plots the loadings $\mathbf{a}_j$ and connects them with the origin. Note that if $r_j > 1$ only the first solution is taken.
\end{itemize} 


\subsection{Simple Homogeneity Analysis}
\label{sec:sha}
The first example is a simple (i.e., no level or rank restrictions, no sets defined) three-dimensional homogeneity analysis for the \code{senate} data set \citep{Ada:02}. The data consists of 2001 senate votes on 20 issues selected by Americans for Democratic Action. The votes selected cover a full spectrum of domestic, foreign, economic, military, environmental and social issues. We tried to select votes which display sharp liberal/conservative contrasts. As a consequence, Democrat candidates have many more ``yes" responses than Republican candidates. A full description of the items can be found in the corresponding package help file. The first column of the data set (i.e., 50 Republicans vs. 49 Democrats and 1 Independent) is inactive and will be used for validation.

\begin{Schunk}
\begin{Sinput}
> library(homals)
> data(senate)
> res <- homals(senate, active = c(FALSE, rep(TRUE, 20)), ndim = 3)